\ifx\mainclass\undefined
\documentclass[cn,11pt,chinese,black,simple]{../elegantbook}
% 微分号
\newcommand{\dd}[1]{\mathrm{d}#1}
\newcommand{\pp}[1]{\partial{}#1}

% FT LT ZT
\newcommand{\ft}[1]{\mathscr{F}[#1]}
\newcommand{\fta}{\xrightarrow{\mathscr{F}}}
\newcommand{\lt}[1]{\mathscr{L}[#1]}
\newcommand{\lta}{\xrightarrow{\mathscr{L}}}
\newcommand{\zt}[1]{\mathscr{Z}[#1]}
\newcommand{\zta}{\xrightarrow{\mathscr{Z}}}

% 积分求和号

\newcommand{\dsum}{\displaystyle\sum}
\newcommand{\aint}{\int_{-\infty}^{+\infty}}

% 简易图片插入
\newcommand{\qfig}[3][nolabel]{
  \begin{figure}[!htb]
      \centering
      \includegraphics[width=0.6\textwidth]{#2}
      \caption{#3}
      \label{no :#1}
  \end{figure}
}

% 表格
\renewcommand\arraystretch{1.5}

% 日期

\newcommand{\homep}[1]{\section*{Problem #1}}
\newcommand{\subhome}[1]{\subsection*{SubProblem #1}}

% MATLAB 代码块环境

% \usepackage{ctex}
\usepackage[numbered,framed]{matlab-prettifier}

\lstset{
  language = octave,
  style = Matlab-editor,
  basicstyle = \mlttfamily,
  escapechar = ",
  mlshowsectionrules = true,
}
% 
% \newenvironment{ocode}{\begin{lstlisting}[language=octave]}{\end{lstlisting}}

\lstnewenvironment{ocode}[1][]{%
    \lstset{language=octave,#1}}{}%

\begin{document}
\fi 
\def\chapname{04mos}

% Start Here
\chapter{MOS 管}


\section{MOS 的基本结构}

MOS 由源极漏极栅极和衬底构成。源漏为 n 型则为 nMOS, 为 p 型则为 pMOS 。

\section{MOS 的平衡能带图}

为了使得费米能级连续,接触处的能带下拉,由于氧化物是刚性下拉,另一侧也下降。

\[\frac{p}{p'} = \exp(\frac{\phi_S}{kT})\] 


\section{理想 MOS 电容}

通过外接电压,使得氧化物能级变平。

通过外加偏置,可以得到积累、耗尽与反型几种状态。

\section{阈值电压}

反型的临界:表面电子浓度与内部空穴浓度相等时的外接电压。

\[\frac{n_i^2}{N_A}\exp \frac{\phi_S}{k T} = N_A\] 

\[\phi_S = k T 2 \log (N_A/n_i)\]

那么 \[V_G = V_{ox} + \phi_S = \frac{Q_S(\phi_S)}{C_{ox}} + \phi_S\]

\section{低于阈值的状态}

由于空穴沿位置上升速度极快,可以看作是一个冲激函数。而积累时 \(\phi_S \approx 0\) 

\[V_G = \frac{Q_S}{C_{ox}}\]

激发的载流子全部存在于界面上,电压全部落在氧化层上,\(V_G \approx V_{ox} = Q_S(\phi_S)/C_{ox}\)。

类似的,反型之后电子浓度也会出现急剧上升,电压的进一步上升基本都会落在 \(V_{ox}\) 上,用于能带弯曲的电压只需一小部分就可以引起极大的电荷变化。

\section{小信号电容}

对于积累时,宽度不变,电荷变化;而耗尽时,类似 PN 结,宽度会发生变化,而电荷保持\footnote{电离杂质}在 \(-N_A\),是一个串联的电容;反型会增加极薄的反型层,空间电荷区基本稳定,但是反型电荷由 \(\delta\) 近似会发生变化。

\[C_S = \frac{\epsilon_0 \epsilon_r}{W}, \text{ where } W \propto \sqrt{V_{bi} - V_A} \]

\section{高于阈值的状态}

电荷随着 \(V_G\) 线性增长,随着 \(V_{th}\) 指数增长。

\[\begin{aligned}
    V_G &= \phi_S + V_{ox}\\
    &= \phi_S + E_{ox} x_{ox} \\
    &= \phi_S - \frac{Q_i + Q_F}{\epsilon_r \epsilon_0} \\
\end{aligned}\]

而阈值电压满足

\[\begin{aligned}
    V_{th} &= 2 \phi_F + E_{ox} x_{ox} \\ 
    &= 2 \phi_F - \frac{Q_i(2\phi_F)+Q_F}{\kappa_{ox} \epsilon_0}
\end{aligned}\]

\[\begin{aligned}
    V_G - V_{th} &= (\phi_S - 2 \phi_F) - \frac{Q_i(\phi_S - Q_i(2 \phi_F))}{\kappa_{ox} \epsilon_0} x_{ox} \\
    &\approx - \frac{Q_F(\phi_S) - Q_F(2 \phi_F)}{\kappa_{ox} \epsilon_0} x_{ox} \\
    &= \frac{Q_i}{\epsilon_r \epsilon_0}
\end{aligned}\] 

小信号电容表示为 

\[C_G = \frac{\dd{Q_{G}}}{\dd{V_{G}}}\]

\[\frac{\dd{V_G}}{\dd{Q_G}} = \frac{Q_S/C_{ox}}{\dd{Q_G}} + \frac{\dd{}\phi_S}{\dd{Q_S}}\]

因此 \[\frac{1}{C_G} = \frac{1}{C_{ox}} + \frac{1}{C_S}\]

对于导通电流,与 \(C_G\) 有关,进一步与 \(C_{ox}\) 有关,可以将二氧化硅做薄以增加 \(C_{ox}\) 。

\[I_{on} = q n v = q v V_G C_G \] 

体校应因子 \(m = 1 + C_S / C_{ox}\) ,越小越好。

\[\phi_S = \frac{C_)}{C_O + C_S}V_G = \frac{V_G}{m}\]

亚阈值斜率 

\[SS = \frac{\dd{\log _{10} I_D}}{V_G} = 60 mV/dec \times m\]

% End Here


\let\chapname\undefined
\ifx\mainclass\undefined
\end{document}
\fi 