\ifx\mainclass\undefined
\documentclass[cn,11pt,chinese,black,simple]{../elegantbook}
% 微分号
\newcommand{\dd}[1]{\mathrm{d}#1}
\newcommand{\pp}[1]{\partial{}#1}

% FT LT ZT
\newcommand{\ft}[1]{\mathscr{F}[#1]}
\newcommand{\fta}{\xrightarrow{\mathscr{F}}}
\newcommand{\lt}[1]{\mathscr{L}[#1]}
\newcommand{\lta}{\xrightarrow{\mathscr{L}}}
\newcommand{\zt}[1]{\mathscr{Z}[#1]}
\newcommand{\zta}{\xrightarrow{\mathscr{Z}}}

% 积分求和号

\newcommand{\dsum}{\displaystyle\sum}
\newcommand{\aint}{\int_{-\infty}^{+\infty}}

% 简易图片插入
\newcommand{\qfig}[3][nolabel]{
  \begin{figure}[!htb]
      \centering
      \includegraphics[width=0.6\textwidth]{#2}
      \caption{#3}
      \label{no :#1}
  \end{figure}
}

% 表格
\renewcommand\arraystretch{1.5}

% 日期

\newcommand{\homep}[1]{\section*{Problem #1}}
\newcommand{\subhome}[1]{\subsection*{SubProblem #1}}

% MATLAB 代码块环境

% \usepackage{ctex}
\usepackage[numbered,framed]{matlab-prettifier}

\lstset{
  language = octave,
  style = Matlab-editor,
  basicstyle = \mlttfamily,
  escapechar = ",
  mlshowsectionrules = true,
}
% 
% \newenvironment{ocode}{\begin{lstlisting}[language=octave]}{\end{lstlisting}}

\lstnewenvironment{ocode}[1][]{%
    \lstset{language=octave,#1}}{}%

\begin{document}
\fi 
\def\chapname{03sbd}

% Start Here
\chapter{肖特基二极管}

也可以叫做金属半导体二极管,具有和 PN 结类似的整流功能。

\section{能带图}

\(\Phi_B\) 称为肖特基势。

描述金属只需要一个参数即 \(E_F\) 来描述。在接触边界需要能级的弯曲。

肖特基结瞬态响应更快,电流承载力更大。

\section{电流}

\[J_T(V_A) = J_{s-m}(0) - J_{s-m}(V_A)\]

而 

\[\begin{aligned}
    J_{s-m} &= q n v \\ 
    &= \int_{E>E'}^\infty -q \text{DOS}(E) f(E)\dd{E} \\
    &= -\frac{ N_S}{2} \exp(-\frac{E_c + E'-V_A - E_F-}{kT}) V q \\ 
    &= -\frac{ N_S}{2} \exp(-\frac{V_{bi} - V_A}{kT}) V q \\ 
    &= I_0 \exp(\frac{V_A}{kT})
\end{aligned}\]

肖特基是多数载流子导电的器件。

% End Here

\let\chapname\undefined
\ifx\mainclass\undefined
\end{document}
\fi 