\ifx\mainclass\undefined
\documentclass[cn,11pt,chinese,black,simple]{../elegantbook}
% 微分号
\newcommand{\dd}[1]{\mathrm{d}#1}
\newcommand{\pp}[1]{\partial{}#1}

% FT LT ZT
\newcommand{\ft}[1]{\mathscr{F}[#1]}
\newcommand{\fta}{\xrightarrow{\mathscr{F}}}
\newcommand{\lt}[1]{\mathscr{L}[#1]}
\newcommand{\lta}{\xrightarrow{\mathscr{L}}}
\newcommand{\zt}[1]{\mathscr{Z}[#1]}
\newcommand{\zta}{\xrightarrow{\mathscr{Z}}}

% 积分求和号

\newcommand{\dsum}{\displaystyle\sum}
\newcommand{\aint}{\int_{-\infty}^{+\infty}}

% 简易图片插入
\newcommand{\qfig}[3][nolabel]{
  \begin{figure}[!htb]
      \centering
      \includegraphics[width=0.6\textwidth]{#2}
      \caption{#3}
      \label{no :#1}
  \end{figure}
}

% 表格
\renewcommand\arraystretch{1.5}

% 日期

\newcommand{\homep}[1]{\section*{Problem #1}}
\newcommand{\subhome}[1]{\subsection*{SubProblem #1}}

% MATLAB 代码块环境

% \usepackage{ctex}
\usepackage[numbered,framed]{matlab-prettifier}

\lstset{
  language = octave,
  style = Matlab-editor,
  basicstyle = \mlttfamily,
  escapechar = ",
  mlshowsectionrules = true,
}
% 
% \newenvironment{ocode}{\begin{lstlisting}[language=octave]}{\end{lstlisting}}

\lstnewenvironment{ocode}[1][]{%
    \lstset{language=octave,#1}}{}%

\begin{document}
\fi 
\def\chapname{05mosiv}

% Start Here
\chapter{MOS I-V 特性}

核心问题是高低状态以及转换特性。

\section{栅压作用}

使得能带沿着电流法向产生弯曲,也可以叫做电掺杂。

\section{漏压作用}

在边角处产生弯曲。 

\section{平方律}

在夹断之前

\[I_{\mathrm{D}}=\frac{Z \bar{\mu}_{\mathrm{n}} C_{\mathrm{o}}}{L}\left[\left(V_{\mathrm{G}}-V_{\mathrm{T}}\right) V_{\mathrm{D}}-\frac{V_{\mathrm{D}}^{2}}{2}\right] \quad\left(\begin{array}{l}
    0 \leqslant V_{\mathrm{D}} \leqslant V_{\mathrm{Dsat}} \\
    V_{\mathrm{G}} \geqslant V_{\mathrm{T}}
\end{array}\right)\]

夹断之后

\[I_{\mathrm{D}_{\mathrm{sat}}}=\frac{Z \bar{\mu}_{\mathrm{n}} C_{\mathrm{o}}}{L}\left[\left(V_{\mathrm{G}}-V_{\mathrm{T}}\right) V_{\mathrm{D}_{\mathrm{Sat}}}-\frac{V_{\mathrm{Dsat}}^{2}}{2}\right]\]

也可以表示为 

\[I_{D|V_D > V_{Dsat}} = I_{D|V_D=V_{Dsat}} \equiv I_{Dsat}\]

当 \(V_D\) 趋近于饱和时,进一步化简:

\[I_{\mathrm{Dsat}}=\frac{Z \bar{\mu}_{\mathrm{n}} C_{\mathrm{o}}}{2 L}\left(V_{\mathrm{G}}-V_{\mathrm{T}}\right)^{2}\]

沟道的调制效应使得电压电流曲线的弯曲。

\section{体电荷}

\[Q_{N}(y)=-C_{o}\left(V_{G}-V_{T}-\phi\right)+q N_{A}\left[W(y)-W_{T}\right]\]

% TODO: 不加下标的 \phi 是啥来着?

\[\begin{aligned}
    W(y) &=\left[\frac{2 K_{\mathrm{S}} \varepsilon_{0}}{q N_{\mathrm{A}}}\left(2 \phi_{\mathrm{F}}+\phi\right)\right]^{1 / 2} \\
    W_{\mathrm{T}} &=\left[\frac{2 K_{\mathrm{S}} \varepsilon_{0}}{q N_{\mathrm{A}}}\left(2 \phi_{\mathrm{F}}\right)\right]^{1 / 2}
\end{aligned}\]

板书:\(V_{th}(V) - V_{th}(0) = V\) , \(V_{Dsat} = (V_{ds} - V_{th}) / m\) ,\(Q_i(V) = C_{ox} (V_G - V_T - m V)\)


积分,得到

\[I_{\mathrm{D}}=\frac{Z \bar{\mu}_{\mathrm{n}} C_{\mathrm{o}}}{L}\left\{\left(V_{\mathrm{G}}-V_{\mathrm{T}}\right) V_{\mathrm{D}}-\frac{V_{\mathrm{D}}^{2}}{2}-\frac{4}{3} V_{\mathrm{W}} \phi_{\mathrm{F}}\left[\left(1+\frac{V_{\mathrm{D}}}{2 \phi_{\mathrm{F}}}\right)^{3 / 2}-\left(1+\frac{3 V_{\mathrm{D}}}{4 \phi_{\mathrm{F}}}\right)\right]\right\}\]

那么饱和电压

\[V_{\mathrm{Dsat}}=V_{\mathrm{G}}-V_{\mathrm{T}}-V_{\mathrm{w}}\left\{\left[\frac{V_{\mathrm{G}}-V_{\mathrm{T}}}{2 \phi_{\mathrm{F}}}+\left(1+\frac{V_{\mathrm{W}}}{4 \phi_{\mathrm{F}}}\right)^{2}\right]^{1 / 2}-\left(1+\frac{V_{\mathrm{W}}}{4 \phi_{\mathrm{F}}}\right)\right\}\]

\section{线性区}

\[v_d = \frac{-\mu E}{(1+(E/E_C)^2)^{1/2}}\]



\section{速度饱和}

仅出现在小尺寸器件。

\[J = nqv = v_{sat} C_{ox} (V_G - V_{th})\] 

可以通过 \(I_D - V_{DS}\) 的关系,判断器件大小。

速度过冲问题:源漏由于电子密度较大,因此电子速度较慢。进入沟道后,未碰撞之前会产生一个较大的速度。

\section{阈值变化}

对于非平带的电压,通过拉平,计算其阈值电压

\[V_{th} = (2 \phi_F - \frac{Q_B}{C_{Ox}}) + V_{FB}\]

\section{氧化层中有电荷}

在氧化层中存在固定电荷。
% End Here


\section{不稳定性}

NaCl 引起的离子不稳定,温度偏压的不稳定性。
\let\chapname\undefined
\ifx\mainclass\undefined
\end{document}
\fi 