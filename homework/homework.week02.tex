\documentclass[lang=cn,11pt,a4paper,cite=authoryear]{elegantpaper}
\usepackage{pifont}
\usepackage{bbding}
% 微分号
\newcommand{\dd}[1]{\mathrm{d}#1}
\newcommand{\pp}[1]{\partial{}#1}

% FT LT ZT
\newcommand{\ft}[1]{\mathscr{F}[#1]}
\newcommand{\fta}{\xrightarrow{\mathscr{F}}}
\newcommand{\lt}[1]{\mathscr{L}[#1]}
\newcommand{\lta}{\xrightarrow{\mathscr{L}}}
\newcommand{\zt}[1]{\mathscr{Z}[#1]}
\newcommand{\zta}{\xrightarrow{\mathscr{Z}}}

% 积分求和号

\newcommand{\dsum}{\displaystyle\sum}
\newcommand{\aint}{\int_{-\infty}^{+\infty}}

% 简易图片插入
\newcommand{\qfig}[3][nolabel]{
  \begin{figure}[!htb]
      \centering
      \includegraphics[width=0.6\textwidth]{#2}
      \caption{#3}
      \label{no :#1}
  \end{figure}
}

% 表格
\renewcommand\arraystretch{1.5}

% 日期

\newcommand{\homep}[1]{\section*{Problem #1}}
\newcommand{\subhome}[1]{\subsection*{SubProblem #1}}

% MATLAB 代码块环境

% \usepackage{ctex}
\usepackage[numbered,framed]{matlab-prettifier}

\lstset{
  language = octave,
  style = Matlab-editor,
  basicstyle = \mlttfamily,
  escapechar = ",
  mlshowsectionrules = true,
}
% 
% \newenvironment{ocode}{\begin{lstlisting}[language=octave]}{\end{lstlisting}}

\lstnewenvironment{ocode}[1][]{%
    \lstset{language=octave,#1}}{}%


\title{微电子器件物理\quad 第二周作业}
\author{范云潜 18373486}
\institute{微电子学院 184111 班}
\date{\zhtoday}

\begin{document}

\maketitle

作业内容:

1,阅读《半导体器件基础》第五章 done

2,使用E5.1和E5.3的程序,画出图E5.1和E5.3 done

3、使用E5.4的程序,画图3种情况下的能带图done

4、修改E5.4的程序,利用subplot函数,使之可以画出掺杂浓度、净电荷、电场、电势随着位置的关系图(类似图5.9)done

5、修改E5.4的程序,使之可以画出施加不同偏压之后的能带图 done

6、课后作业:5.1、5.2、5.3、5.4

\tableofcontents

% Start Here

\homep{E5.1 E5.3}

仿真图片分别 \figref{no :fig1},\figref{no :fig2} 。

\lstinputlisting{homework02/e5_1.m}

\lstinputlisting{homework02/e5_3.m}

\lstinputlisting{homework02/e5_4.m}

\qfig[fig1]{homework02/hw02p1.pdf}{题1图}

\qfig[fig2]{homework02/hw02p2.pdf}{题2图}

\homep{E5.4}

\homep{5.1}

a) \XSolidBrush % why
 b) \Checkmark c) \XSolidBrush d)\Checkmark e)\Checkmark f)\XSolidBrush g) \Checkmark h) \Checkmark i) \XSolidBrush j) \Checkmark 

\homep{5.2}

\subhome{a}

\[p = N_V \exp(\frac{E_i-E_F}{kT})\]

如 \figref{no :01} 。
\qfig[01]{h201.png}{5.2 能带}

\subhome{b}

\[E_F = E_V - 2 k T = E_C - E_G / 4\]

\[E_C + V_{bi} q = E_V + E_G\]

联立

\[V_{bi} = \frac{3}{4}E_G + 2 k T\]

\homep{5.3}

\subhome{a}


如 \figref{no :02} 。
\qfig[02]{h202.png}{5.3 能带}

\subhome{b}

内建电势满足

\[V_{bi} q = E_{V1} - E_{V2} \]

而 

\[N_{A1} = p = N_V \exp (- \frac{E_i - E_{V1}}{kT})\]

\[N_{A2} = p = N_V \exp (- \frac{E_i - E_{V2}}{kT})\]

\[\frac{N_{A1}}{N_{A2}} = \exp(\frac{V_{bi}q}{kT})\]

那么 

\[V_{bi} = \log(\frac{N_{A1}}{N_{A2}}) \frac{kT}{q}\]

\subhome{c}

如\figref{no :03} 。
\qfig[03]{h203.png}{5.3 电场、电势、电荷}

\subhome{d}

耗尽近似对结区只考虑电离杂质,对远离结区部分认为其平衡,总电荷密度为 0 。

\subhome{e}

静电变量表示为: 

\[\rho =  q (p - n + N_D - N_A)\]


如\figref{no :04} 。
\qfig[04]{h204.png}{5.3 静电变量}

在远离结区的部分:\(\rho = q \cdot p\)

在结区: \(\rho = q (p - N_A)\)


不适用于耗尽近似,因为在结区没有发生两种载流子的耗尽。

\homep{5.4}


\[V_{bi} = \frac{kT}{q} \log(\frac{N_A N_D}{n_i^2}) = 0.61316 V\] 


\[x_p = \left[\frac{2 K_S \epsilon_0}{1} \left(\frac{N_D}{N_A(N_A+N_D)}\right)\right] = 0.000073002 cm\]

\[x_p = \left[\frac{2 K_S \epsilon_0}{1} \left(\frac{N_A}{N_D(N_A+N_D)}\right)\right] = 0.000036501 cm\]

\[W = x_n + x_p = \left[\frac{2 K_S \epsilon_0}{1} \left(\frac{N_A + N_D}{N_AN_D}\right)\right]^{1/2} = 0.00010950 cm \]


\[E(0) = - \frac{q N_D}{K_S \epsilon_0} (x_n) = 11198.88614 V/cm \]

\[V(0) = - \frac{q N_A}{K_S \epsilon_0} \frac{1}{2} x_p^2 = 0.20439 V\]
% End Here

\end{document}