\ifx\mainclass\undefined
\documentclass[cn,11pt,chinese,black,simple]{../elegantbook}
% 微分号
\newcommand{\dd}[1]{\mathrm{d}#1}
\newcommand{\pp}[1]{\partial{}#1}

% FT LT ZT
\newcommand{\ft}[1]{\mathscr{F}[#1]}
\newcommand{\fta}{\xrightarrow{\mathscr{F}}}
\newcommand{\lt}[1]{\mathscr{L}[#1]}
\newcommand{\lta}{\xrightarrow{\mathscr{L}}}
\newcommand{\zt}[1]{\mathscr{Z}[#1]}
\newcommand{\zta}{\xrightarrow{\mathscr{Z}}}

% 积分求和号

\newcommand{\dsum}{\displaystyle\sum}
\newcommand{\aint}{\int_{-\infty}^{+\infty}}

% 简易图片插入
\newcommand{\qfig}[3][nolabel]{
  \begin{figure}[!htb]
      \centering
      \includegraphics[width=0.6\textwidth]{#2}
      \caption{#3}
      \label{#1}
  \end{figure}
}

% 表格
\renewcommand\arraystretch{1.5}

% 日期

\newcommand{\homep}[1]{\Large\textbf{Problem #1}}
\newcommand{\subhome}[1]{\large\textbf{SubProblem #1}}

% MATLAB 代码块环境

% \usepackage{ctex}
\usepackage[numbered,framed]{matlab-prettifier}

\lstset{
  language = octave,
  style = Matlab-editor,
  basicstyle = \mlttfamily,
  escapechar = ",
  mlshowsectionrules = true,
}
% 
% \newenvironment{ocode}{\begin{lstlisting}[language=octave]}{\end{lstlisting}}

\lstnewenvironment{ocode}[1][]{%
    \lstset{language=octave,#1}}{}%

\begin{document}
\fi 
\def\chapname{00intro}

% Start Here
\chapter{绪论}

微电子硅基芯片的尺度即将到达瓶颈,是否可能存在进一步改进的可能,是否可以找到全新的替代材料?

\section{半导体物理回顾}

晶体结构部分

\begin{itemize}
    \item 了解并使用密勒指数
    \item 理解基本的晶体结构
    \item 理解量子力学的基本知识
\end{itemize}

经典的理论只能处理单电子的问题,甚至 \ce{He_2} 已经无法处理,而实际的固体存在 \(10^{-23} \text{cm}^{-3}\) 的电子密度。进一步发现了晶体的周期性,三维重复性对问题进行了简化。对硅片来说,晶圆的方向就是 \([1 0 0]\) 晶向。当晶体纯度足够高时,我们就可以在单个的原胞内部处理。

当硅原子相隔较远,电子不发生交叠,均处在 \(E_1\) 能级,若是逐渐接近,能量会变化为 \(E \pm \Delta E'\) 的能带分布。若是有 \(N\) 个电子,不考虑自旋兼并度就有 \(N\) 个能带,考虑则是 \(2 N\) 个。

假设我们有一块 \(N\) 原子晶体,那么能量以及相互作用的矩阵可以表示为

\[
    \left[
        \begin{matrix}
            E_1 & E_{12} & E_{13} & & \\
            E_{21} & E_2 & E_{23} & &  \\
            E_{31} & E_{32} & E_{3} & & \\
            & & & \ddots &
        \end{matrix}    
    \right]
\]

但是特征值的求解几乎是不可能的,那么我们使用原胞的思想就可以块对角化,求解特征值对应的能量:


\[
    \left[
        \begin{matrix}
            [4 \times 4] & & & & \\
            & [4\times 4] & & & \\
            & & [4\times 4] & & \\
            & & & \ddots &
        \end{matrix}    
    \right]
\]

实际上动量以及角动量对应着平移不变性以及旋转不变性。得到的 \(E-K\) 关系就是能量-动量关系。

{关键方程}

\begin{itemize}
    \item 物理常数
    \item 密勒指数
    \item 晶面角度
    \item 立方晶面距离
\end{itemize}

\subsection{能带结构}

\begin{itemize}
    \item 理解 \(E-K\) 关系,从中提取速度、有效质量、等能面等信息,进一步获得电流等关键量
    \item 理解态密度,费米积分,并进行简单计算
    \item 一定温度以及掺杂下,计算电子空穴浓度
\end{itemize}

布洛赫波函数为 \(\varphi = \phi (\vc{r}) e^{i \vc{k} \cdot \vc{r}}\)。实际中电子以波包的形式传播,其速度为群速度,与 \(E-K\) 关系斜率相关(色散关系的倒数)。

综上,我们从真空电子\textbf{概率波}与晶格作用(周期势阱)得到了 \(E-K\) 关系,导出了有效质量以及群速度等关系,转换为另一个\textbf{粒子}的状态,简化问题。

若是需要求解电流密度 \(\vc{J} = i \cdot n \cdot \vc{v}\) ,其中\(n\)总满足\eqref{eq:00:01},为所有量子态的占据几率,其中 \(N(E)\) 为态密度。

\begin{equation}\label{eq:00:01}
    n = \dsum_{k = 0}^{n = N} f\left(E(\vc{k})\right) = A \int_{\vc{k}} f\left(E(\vc{k})\right) \dd{\vc{k}^3} A' \int_{E=0}^\infty f\left(E\right) N(E) \dd{E} 
\end{equation}

最终得到

\[
\begin{aligned}
    n &= N_C e^{\frac{E_F-E_C}{k T}} \\
    p &= N_C e^{\frac{E_V-E_F}{k T}} \\
\end{aligned}
\]

那么 \(n \cdot p = N_C N_V e^{-\frac{E_g}{k T}} = n_i^2\) 其中 \(E_g = E_C - E_V\) 。

无掺杂时, \(n = p = n_i \approx 10^{10} \text{cm}^{-3}\) 近似绝缘。有掺杂时,电中性条件为 

\begin{equation}\label{eq:00:02}
    p - n + N_D - N_A = 0
\end{equation}


在导带或者价带能量与费米能级相差\(3 k_B T\) 及以上时, 费米-狄拉克分布可以弱化为玻尔兹曼分布,并且可以称为简并半导体。

关键方程

\begin{itemize}
    \item 不同维度的能量、动量态密度,
    \item 费米狄拉克分布
    \item 费米积分
\end{itemize}

\subsection{非平衡态}

\begin{itemize}
    \item 理解载流子的产生与复合
    \item 理解载流子的扩散与漂移
    \item 初步求解简单的少数载流子扩散方程
\end{itemize}

虽然不满足 \(n p = n_i^2\) 但是仍均匀分布在空间中。当出现扰动时,偏离稳定,通过耗散能量逐渐回到平衡态。由于电子空穴成对出现,那么在恢复稳态时有

\[
(n \pm \Delta n) \times (p \pm \Delta p) = n_i^2, \text{ where } \Delta n = \Delta p
\]

复合中心辅助的产生与复合(SRH)

\begin{equation}
    - \frac{\pp{n}}{\pp{t}}_{SRH} = - \frac{\pp{p}}{\pp{t}}_{SRH} = R_{SRH} = \frac{np - n_i^2}{\tau_p(n + n_1) + \tau_n(p + p_1)}
\end{equation}


\(D_P, D_N\)分别是空穴、电子的扩散系数,单位是 \(\text{cm}^2/\text{sec}\)

\[\begin{array}{l}
    \left. J _{ P }\right|_{\text {diff }}=-q D_{ P } \nabla p \\
    J _{ N \mid \text { diff }}=q D_{ N } \nabla n
\end{array}\]


考虑浓度差

\begin{equation*}
    \begin{array}{l}
        J _{N}=q n \mu_{N} E+q D_{N} \nabla n \\
        J _{P}=q p \mu_{P} E-\left(q D_{P} \nabla p\right) 
    \end{array}
\end{equation*}

第一项是漂移电流,第二项是扩散电流。


漂移电流的定义 

\[J_n = q n v_{drift} = q n \mu_n \mathscr{E} = n \mu_n \frac{\dd{F_n}}{\dd{x}}\]

\[J_p = q p v_{drift} = q p \mu_p \mathscr{E} = p \mu_p \frac{\dd{F_p}}{\dd{x}}\]

其中 \(\mu_n\) 是迁移率。

关键方程

\begin{itemize}
    \item 半导体方程:连续性方程与泊松方程
    \item 少数载流子扩散方程
\end{itemize}


% End Here
% 
\let\chapname\undefined
\ifx\mainclass\undefined
\end{document}
\fi 