\ifx\mainclass\undefined
\documentclass[cn,11pt,chinese,black,simple]{../elegantbook}
% 微分号
\newcommand{\dd}[1]{\mathrm{d}#1}
\newcommand{\pp}[1]{\partial{}#1}

% FT LT ZT
\newcommand{\ft}[1]{\mathscr{F}[#1]}
\newcommand{\fta}{\xrightarrow{\mathscr{F}}}
\newcommand{\lt}[1]{\mathscr{L}[#1]}
\newcommand{\lta}{\xrightarrow{\mathscr{L}}}
\newcommand{\zt}[1]{\mathscr{Z}[#1]}
\newcommand{\zta}{\xrightarrow{\mathscr{Z}}}

% 积分求和号

\newcommand{\dsum}{\displaystyle\sum}
\newcommand{\aint}{\int_{-\infty}^{+\infty}}

% 简易图片插入
\newcommand{\qfig}[3][nolabel]{
  \begin{figure}[!htb]
      \centering
      \includegraphics[width=0.6\textwidth]{#2}
      \caption{#3}
      \label{no :#1}
  \end{figure}
}

% 表格
\renewcommand\arraystretch{1.5}

% 日期

\newcommand{\homep}[1]{\section*{Problem #1}}
\newcommand{\subhome}[1]{\subsection*{SubProblem #1}}

% MATLAB 代码块环境

% \usepackage{ctex}
\usepackage[numbered,framed]{matlab-prettifier}

\lstset{
  language = octave,
  style = Matlab-editor,
  basicstyle = \mlttfamily,
  escapechar = ",
  mlshowsectionrules = true,
}
% 
% \newenvironment{ocode}{\begin{lstlisting}[language=octave]}{\end{lstlisting}}

\lstnewenvironment{ocode}[1][]{%
    \lstset{language=octave,#1}}{}%

\begin{document}
\fi 
\def\chapname{01pnjunction}

% Start Here
\chapter{PN 结}

\begin{introduction}
    \item 什么是 PN 结
    \item 平衡态能带图
    \item 外加电压
\end{introduction}

PN 结是其他微电子器件的基础。

PN 结的常见应用有:太阳能电池,GaAs / GaN 激光器,有机发光二极管,雪崩光电二极管,CMOS 图像传感器。OLED 是有机半导体的显示材料,怕水,寿命有限。

\section{PN 结的形成}

PN 结最早是通过热扩散,目前有沉积、扩散、激光掺杂等工艺。

一般掺杂得到的是二维的器件,电流方向不是直线,难以求解。若是得到的是一个很窄的器件,可以简化为一个一维问题,对第二个维度的依赖性会变低,只需考虑第一维度的运动情况。

两种半导体直接相连是不能得到 PN 结的,因为断面上的原子不能形成化学键。对 N 型半导体,费米能级靠近导带,对于 P 型半导体,靠近价带。在两个体区中, N 型中很多的施主杂质电离的正离子(无法移动),但是有等量的电子(可以移动, \(n = N_D\));类似的 P 型存在可以移动的空穴(\(p = N_A\))。浓度差造成了扩散,那么 N 型靠近结区的部分电子被中和,整体带正电,P 型对应部分带负电,这就是耗尽区。由于扩散的存在,出现 \(n \cdot p > n_i^2\) 的瞬态。

电流分为两部分,漂移电流以及扩散电流。之前考虑的是扩散电流,因为此时还没有出现两部分的电势差。由于扩散后出现了静电荷,形成了电场,开始考虑漂移电流。电场逐渐增大,受到的阻力也越来越大,直到扩散电流与漂移电流相互抵消,进入稳态,即完成形成过程。

可以分为两个体区以及耗尽区(空间电荷区),耗尽区中没有自由移动的电荷, N 型一侧有 \(n \ll n_0\),另一侧有 \(p \ll p_0\) 。

\(E_c(x)\) 如\figref{01pnjunction:fig0201}

\qfig[fig0201]{0201.png}{\(E_c(x)\)变化}

\(n(x)\) 满足 \[n(x) = N_c \exp(-\frac{E_c(x)-E_F}{k T})\] 

其中, \(E_c - E_F \approx 0.06 eV\), \(k T \approx 26 meV\)。

依靠本式,得到,空间电荷区几乎没有自由电荷。最后的分布如\figref{01pnjunction:fig0202}

\qfig[fig0202]{0202.png}{PN 结区域}

形成的电场为均匀变化,最大值为 

\[E_{max} = \frac{q}{K_s \epsilon_0} x_p N_A = \frac{q}{K_s \epsilon_0} x_n N_D\]

可以得到对应的电势以及能带图。

\section{平衡态能带图}

\qfig[fig0203]{0203.png}{电场\(E\)变化}

分析空间电荷区电势在施主区的能级:

\[N_D = n = N_C \exp (\frac{E_i-E_C}{kT})\]

解得 

\[\log (\frac{n}{N_C}) = - \frac{E_C - E_i}{kT}\]

那么

\[E_C = - k T \log(\frac{n}{N_C}) + E_i\]

同理

\[N_A = p = N_V \exp (- \frac{E_i - E_V}{kT})\]

\[E_V = E_i + k T \log(\frac{p}{N_V}) \]

带隙为 \[E_g = E_C - E_V = - k T \log(\frac{n_i^2}{N_C N_V})\]

而 

\[np = N_C N_V \exp (- \frac{E_g}{kT})\]

\[E_g = -k T\log(\frac{np}{N_C N_V})\]

内建电势满足 \[V_{bi} = E_g - (E_F - E_V) - (E_{C,otherside} - E_F)\]

\[V_{bi} =  = V_g + kT\log(\frac{N_C}{N_D}) + kT \log (\frac{N_V}{N_A}) = \frac{kT}{q} \log(\frac{N_A N_D}{n_i^2})\]


\section{耗尽近似:泊松方程解析解}


泊松方程是静电变量的出发点,适用于三维的公式为

\[\nabla \cdot E = \frac{\rho}{K_S \epsilon_0}\]

杂质分布为,其中 \(\rho = q (p - n + N_D - N_A)\)

\[
    \begin{aligned}
        \frac{\dd{E}}{\dd{x}} = - \frac{\rho}{\epsilon_r} = \frac{\rho}{K_s \epsilon_0} 
    \end{aligned}
\]

那么 \[\begin{aligned}
    E(x) =& \int - \frac{q N_D}{\epsilon_r} \dd{x} \\
    =& -\frac{q N_D}{\epsilon_r} x
\end{aligned}\]


多次掺杂需要浓度的提升。

\section{耗尽区的长度}

在突变近似以及耗尽近似下,

\[\frac{\dd{E}}{\dd{x}} = \left\{\begin{aligned}
    - q N_A / K_S \epsilon_0, &\quad -x_p \leq x \leq 0 \\
    q N_D / K_S \epsilon_0, &\quad 0 \leq x \leq x_n \\
    0, &\quad else 
\end{aligned}\right.\]

得到 

\[E(x) = \left\{
\begin{aligned}
    - \frac{q N_A}{K_S \epsilon_0} (x_p + x), &\quad -x_p \leq x \leq 0 \\
    - \frac{q N_D}{K_S \epsilon_0} (x_n - x), &\quad 0 \leq x \leq x_n 
\end{aligned}    
\right.\]

为了满足\(x = 0\)的电场连续性,得到 

\[N_A x_p = N_D x_n\]

求解电势,电势满足 \(E = -\dd{V}/\dd{x}\)

设置 p 型区边缘的电势为参考电势 \(0\) ,那么在 n 型区边缘的电势为 \(V_{bi}\),那么交界处的电势一致:

\[V_{bi} - \frac{q N_D}{2 K_S \epsilon_0}x_n^2 = \frac{q N_A}{2 K_S \epsilon_0}x_p^2\]

得到

\[x_p = \left[\frac{2 K_S \epsilon_0}{1} \left(\frac{N_D}{N_A(N_A+N_D)}\right)\right]\]
\[x_p = \left[\frac{2 K_S \epsilon_0}{1} \left(\frac{N_A}{N_D(N_A+N_D)}\right)\right]\]
\[W = x_n + x_p = \left[\frac{2 K_S \epsilon_0}{1} \left(\frac{N_A + N_D}{N_AN_D}\right)\right]^{1/2}\]

\section{外加电压}

正向偏置是指的增加正向电压,使得正电压一端中性区费米能级下降,引起导带价带下降,这样也引起了 \(V_{bi}\) 的下降(平移)。费米能级下降的幅度为外加电压的大小。

\[V_{bi}' = V_{bi} - V_A\]

对原有的耗尽宽度的改进可以直接将 \(V_{bi}\) 替换为 \(V_{bi}'\) 。

若是反向,在 n 型区使得费米能级降低,内建电势 \(V_{bi}\) 会上升,其峰值电场强度会明显增强。

反向偏置的 PN 结可以作为电容。

PN 结的连续性方程:

\[
    \begin{aligned}
        \nabla \cdot E&=q\left(p-n+N_{D}^{+}-N_{A}^{-}\right.\\
        \frac{\partial n}{\partial t}&=\frac{1}{q} \nabla \bullet \mathrm{J}_{N}-r_{N}+g_{N}\\
        \mathbf{J}_{N}&=q n \mu_{N} E+q D_{N} \nabla n\\
        \frac{\partial p}{\partial t}&=\frac{1}{q} \nabla \cdot \mathbf{J}_{p}-r_{P}+g_{P}\\
        \mathbf{J}_{P}&=q p \mu_{P} E-q D_{P} \nabla p
        \end{aligned}
\]

添加外场后(正向),如 ~\figref{\chapname :0204}。由于结区较短,产生复合未达到平衡,因此不满足 \(n p = n_i^2\),但是电子 / 空穴内部达到平衡,因此产生一种\textbf{准费米能级},分别描述两种粒子的状态。\footnote{子系统平衡时间为 ps ,整体平衡为 \(\mu\)s - ms 的量级} 平衡时,电子空穴的准费米能级是重合的。\figref{\chapname :0205}。

\qfig[0204]{0204.png}{外加电压后的能级图}

\qfig[0205]{0205.png}{外加电压后的准费米能级图}

假设外加直流电压,在中性区中没有产生与复合,且 \(n\) 为常数,因此 \(\nabla \cdot \mathbf{J} = 0\) 。

对于 N 区空穴来说,浓度极低,只需考虑 \(J_{P} = - q D_p \nabla p\) ,需要注意 \(p\) 的梯度,也就是只有扩散电流。而电流处处相等,另一区的总电流总是和这一侧的扩散相等的。

那么总电流

\[I = J_{P(N)} + J_{N(N)} = J_{P(n),D} + J_{N(n),D,D} = J_{P(n), D} + J_{N(p),D}\]

假设电流是常数,那么 \(\nabla \cdot J = 0\) :

又

\[J_N = q D_N \nabla \cdot n, \text{ where } n = n_0 + \Delta n\]

\[\nabla^2 \cdot \Delta n = 0\]

因此 

\[\Delta n = A \cdot x + B\]

由边界条件,坐标系如\figref{\chapname :0206}

\qfig[0206]{0206.png}{坐标系建立}

\[\Delta n = A(x - L) \]

\[n(0) = N_C \exp(- \frac{E_C(0) - E_F}{k T}), \text{ where } E_C(0) = E_C + V_{bi} - V_A\]

那么

\[n(0) = N_C \exp(-\frac{E_C-E_F}{kT}) \exp(-\frac{V_{bi}}{kT}) \exp(\frac{V_A}{kT}) = \frac{n_i^2}{p}\exp(\frac{V_A}{kT}) = n_{0,p}\exp(\frac{V_A}{kT})\]

那么 

\[\Delta n = n_{0,p}(\exp(\frac{V_A}{kT}) - 1)\]

假设存在一个线性的变化,到电极处 \(\Delta n = 0\) 存在一个线性的改变。


在无偏置时,扩散电子以及漂移电子可以动态平衡。正向偏置后漂移电子增多,扩散电子减少。

根据结区的电子浓度以及平衡区的电子浓度计算分布,并进行线性近似

正向偏置有 

\[\ln J_T \approx q V_A / k_B T + \ln const\]

\[J_T = -q [\frac{D_n}{W_p}\frac{n_i^2}{n_A} + \frac{D_p}{W_n} \frac{n_i^2}{N_D}] (e^{qV_A \beta} - 1)\]

反向偏置有

\[J_T \approx const\]

反向偏置时:

\[n' = N_C \exp(-\frac{E_C - E_F}{kT}) = N_C \exp(-\frac{E_O + V_{bi} - V_A - E_F}{kT})\]

结区有

\[\frac{N_D N_A}{n_i^2} = \exp(\frac{V_{bi}}{kT}) \frac{n_i^2}{N_A}\]
% End Here


那么 

\[n' = \frac{n_i^2}{N_A} \exp(\frac{V_A}{kT})\]


\section*{\(V_{bi}\) 推导}

由于 

\[n = N_C \exp(- \frac{E_C - E_F}{k T})\] 

对于 \(E_C\) 最低处为 \(n_1\) 另一侧为 \(n_2\) ,那么

\[
\begin{aligned}
    n_1 &= N_D \\
    n_2 &= \frac{n_i^2}{N_A} \\
    \frac{n_1}{n_2} &= \exp(\frac{E_{C,2} - E_{C,1}}{k T}) \\
    \log (N_A N_D / n_i^2) &= \frac{V_{bo}}{k T}
\end{aligned}    
\]
% End Here

\let\chapname\undefined
\ifx\mainclass\undefined
\end{document}
\fi 