\ifx\mainclass\undefined
\documentclass[cn,11pt,chinese,black,simple]{../elegantbook}
% 微分号
\newcommand{\dd}[1]{\mathrm{d}#1}
\newcommand{\pp}[1]{\partial{}#1}

% FT LT ZT
\newcommand{\ft}[1]{\mathscr{F}[#1]}
\newcommand{\fta}{\xrightarrow{\mathscr{F}}}
\newcommand{\lt}[1]{\mathscr{L}[#1]}
\newcommand{\lta}{\xrightarrow{\mathscr{L}}}
\newcommand{\zt}[1]{\mathscr{Z}[#1]}
\newcommand{\zta}{\xrightarrow{\mathscr{Z}}}

% 积分求和号

\newcommand{\dsum}{\displaystyle\sum}
\newcommand{\aint}{\int_{-\infty}^{+\infty}}

% 简易图片插入
\newcommand{\qfig}[3][nolabel]{
  \begin{figure}[!htb]
      \centering
      \includegraphics[width=0.6\textwidth]{#2}
      \caption{#3}
      \label{#1}
  \end{figure}
}

% 表格
\renewcommand\arraystretch{1.5}

% 日期

\newcommand{\homep}[1]{\Large\textbf{Problem #1}}
\newcommand{\subhome}[1]{\large\textbf{SubProblem #1}}

% MATLAB 代码块环境

% \usepackage{ctex}
\usepackage[numbered,framed]{matlab-prettifier}

\lstset{
  language = octave,
  style = Matlab-editor,
  basicstyle = \mlttfamily,
  escapechar = ",
  mlshowsectionrules = true,
}
% 
% \newenvironment{ocode}{\begin{lstlisting}[language=octave]}{\end{lstlisting}}

\lstnewenvironment{ocode}[1][]{%
    \lstset{language=octave,#1}}{}%

\begin{document}
\fi 
\def\chapname{02pnjvi}

% Start Here
\chapter{PN 结交流特性}

电容的本质是电荷随电压变化。 PN 结也是电容,并且是一个可变电容。对 PN 结来说结电容占主导,扩散电容较少。

\section{正向导通电导}

\[I = I_0 (\exp(q(V_A-R_S I)\beta / m) - 1) \]

其中 \(m\) 是非理想因子。

\[\ln \frac{I+I_0}{I_0} = q(V_A - R_S I) \frac{\beta}{m}\]

\[\frac{1}{R} = \frac{\dd{I_A}}{\dd{V_A}}\]

\section{结电容}

结区宽度满足

\[W \propto \sqrt{V_{bi} - V_A}\]

那么随着小信号的增大,空间电荷区变窄,p 区费米能级下降;
减小则空间电荷区变宽,p 区费米能级上升。

那么外加小信号 \(V_{AC}\) 后势垒也随之变化,


计算流程为

\[V_{AC} \rightarrow \Delta V_{bi} \rightarrow \Delta W \rightarrow \Delta Q = \Delta W_n N_D\]

\[C = \frac{\Delta Q}{V_{AC}}\]

若是看成平板电容器

\[C = \frac{\epsilon_r \epsilon_0 A}{(W_n + W_p)}\] 

\(A\) 是结面积。由于宽度由直流偏置决定,因此这是一个可变电容。


\section{多数载流子}

运动速度为 ps 量级,决定结区的性质。

\section{内建电势}

\[\frac{1}{C_j^2} \approx \frac{2}{q N_D(x) K_s \epsilon_0 A^2}(V_{bi} - V_A)\]


\section{少数载流子}

少数载流子决定扩散电容。

\[J_N = q n \mu_N E + q D_N \frac{\dd{n}}{\dd{x}}\]

\[\frac{\pp{n}}{\pp{t}} = \frac{1}{q}\frac{\dd{J_n}}{\dd{x}}-r_N + g_N\]

\[\frac{\pp{(n_0 + \Delta n_{dc} \Delta n_{ac} e^{j \omega t})}}{\pp{t}} = D_N \frac{\dd{^2 (n_0 + \Delta n_{dc} \Delta n_{ac} e^{j \omega t})}}{\dd{x^2}} - \frac{\Delta n_{dc} + \Delta n_{ac} e^{j\omega t}}{\tau_n}\]

少数载流子的变化在微秒级。


\section{数字电路中的应用}

一般来说,在电压突变时,会发生电压保持、电流突变的行为来维持电容特性。

\let\chapname\undefined
\ifx\mainclass\undefined
\end{document}
\fi 