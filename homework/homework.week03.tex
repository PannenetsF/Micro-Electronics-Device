\documentclass[lang=cn,11pt,a4paper,cite=authoryear]{elegantpaper}

% 微分号
\newcommand{\dd}[1]{\mathrm{d}#1}
\newcommand{\pp}[1]{\partial{}#1}

% FT LT ZT
\newcommand{\ft}[1]{\mathscr{F}[#1]}
\newcommand{\fta}{\xrightarrow{\mathscr{F}}}
\newcommand{\lt}[1]{\mathscr{L}[#1]}
\newcommand{\lta}{\xrightarrow{\mathscr{L}}}
\newcommand{\zt}[1]{\mathscr{Z}[#1]}
\newcommand{\zta}{\xrightarrow{\mathscr{Z}}}

% 积分求和号

\newcommand{\dsum}{\displaystyle\sum}
\newcommand{\aint}{\int_{-\infty}^{+\infty}}

% 简易图片插入
\newcommand{\qfig}[3][nolabel]{
  \begin{figure}[!htb]
      \centering
      \includegraphics[width=0.6\textwidth]{#2}
      \caption{#3}
      \label{no :#1}
  \end{figure}
}

% 表格
\renewcommand\arraystretch{1.5}

% 日期

\newcommand{\homep}[1]{\section*{Problem #1}}
\newcommand{\subhome}[1]{\subsection*{SubProblem #1}}

% MATLAB 代码块环境

% \usepackage{ctex}
\usepackage[numbered,framed]{matlab-prettifier}

\lstset{
  language = octave,
  style = Matlab-editor,
  basicstyle = \mlttfamily,
  escapechar = ",
  mlshowsectionrules = true,
}
% 
% \newenvironment{ocode}{\begin{lstlisting}[language=octave]}{\end{lstlisting}}

\lstnewenvironment{ocode}[1][]{%
    \lstset{language=octave,#1}}{}%


\title{微电子器件物理\quad 第三周作业}
\author{范云潜 18373486}
\institute{微电子学院 184111 班}
\date{\zhtoday}

\begin{document}

\maketitle

作业内容:pn结作业(见hw03-pnjunction.pdf)

\homep{1}

\subhome{a}

\[V_{bi} = \frac{kT}{q}\ln \frac{N_A N_D}{n_i^2} =  0.59524 V\]

\subhome{b}

\[x_p = x_n = \left[\frac{2 K_s \epsilon_0}{q} \frac{N_A}{N_D(N_A + N_D)}\right]^{1/2} = 0.000062291 cm\] 

\[W = x_n + x_p = 0.00012458 cm\]

\subhome{c}

\[V(x = 0) = \frac{q N_A}{2K_S \epsilon_0} x_p^2 = 0.29762 V\]

\[E(x=0) = - \frac{q N_A}{K_S \epsilon_0} x_p = -9555.8 V/cm\]

\subhome{d}

如 \figref{1d} 

\qfig[1d]{h301.png}{草图}

\homep{2}

\subhome{a}

\[V_{bi} = \frac{kT}{q}\ln \frac{N_A N_D}{n_i^2}\]

\[W \approx \left[\frac{2K_S\epsilon_0}{q} \frac{1}{N_D}V_{bi}\right]\]

\[E(0) = - \frac{q N_A}{K_S \epsilon_0} \left[\frac{2K_S \epsilon_0}{q}\frac{N_D}{N_A^2}\right]^{1/2}\]

\[V(x) = \left\{ \begin{aligned}
    &\frac{q N_A}{2 K_S \epsilon_0} (x+[\frac{2K_S\epsilon_0}{q} \frac{N_D}{N_A^2}V_{bi}]^{1/2})^2  \\
    &V_{bi}-\frac{q N_D}{2 K_S \epsilon_0} (-x+[\frac{2K_S\epsilon_0}{q} \frac{1}{N_D}V_{bi}]^{1/2})^2 
\end{aligned} \right.\]

\[\rho(x) = -q N_A, \text{ if } -x_p \leq x \leq 0; 0 \text{ else}\]

\subhome{b}


\[V_{bi} = \frac{kT}{q}\ln \frac{N_A N_D}{n_i^2} =  0.83334V\]

\subhome{c}

\[x_n = \left[\frac{2 K_s \epsilon_0}{q} \frac{N_A}{N_D(N_A + N_D)}\right]^{1/2} = 0.00010423 cm\]

\[x_p = \frac{N_D}{N_A} x_n = 0.000000010423 cm\]

\[W = x_n + x_p = 0.00010424 cm\]

\subhome{d}

\[V(x=0) = \frac{q N_A }{2 K_S \epsilon_0} x_p^2 = 0.000083325 V\]

\[E(x=0) = - \frac{qN_A}{K_S \epsilon_0}x_p = -15989.04177 V/cm\]

\subhome{e}

如 \figref{2e}

\qfig[2e]{h302.png}{草图}

\homep{3}

\subhome{a}

\[\rho = \left\{
\begin{aligned}
    q N_D , & -x_n \leq x \leq 0\\
    0, & 0 \leq x \leq x_i\\ 
    -q N_A, & x_i \leq x \leq x_i + x_p 
\end{aligned}    
\right.\]

如 \figref{3a},由于掺杂浓度差距过大,图中大小仅作示意,不为真实大小。

\qfig[3a]{h303.png}{电场}

电场大小表示为 

\[
E = \left\{
\begin{aligned}
    \frac{q N_D}{K_S \epsilon_0} (x+x_n), & -x_n \leq x \leq 0 \\
    \text{const}, & 0 \leq x \leq x_i \\ 
    -\frac{q N_A}{K_S \epsilon_0} (x - x_i - x_p), & x_i \leq x \leq x_i + x_p
\end{aligned}    
\right.    
\]

连续性得到:\(x_n N_D = x_p N_P\) 

设在 P 侧中性区电势为 \(0\) ,那么 N 侧的中性区电势为 \(V_{bi}\) 。由于 \(x_p > x_i\) 近似忽略后者。

\[V_{bi} = \frac{1}{2} (x_i + x_i + x_n + x_p) \cdot \frac{q N_A x_p}{K_S \epsilon_0} \approx (\frac{N_A + N_D}{2 N_D}) \frac{q N_A}{K_S \epsilon_0} x_p^2 \approx \frac{N_A q}{2 K_S \epsilon_0} x_p^2 \]

解得 \(x_p = \sqrt{\frac{2 K_S \epsilon_0 V_{bi}}{N_A q}}\)

\subhome{c}

由电场表示\figref{3a}可知本征层的存在扩展了电场最大值的空间,增大了 \(V_{bi}\)

\subhome{d}

最大电场大小没有变化,但是分布变广了。

\homep{4}

\[I_0 = q A (\frac{D_N}{L_N} \frac{n_i^2}{N_A} + \frac{D_p}{L_p}\frac{n_i^2}{N_D})\]

\[L = \sqrt{\tau D}, \frac{D}{\mu} = \frac{k T}{q}\]

\[I = I_0 (\exp(a V_A / k T) - 1)\]

带入 MATLAB 

\begin{lstlisting}
T=300;        % Temperature in Kelvin
k=8.617e-5;   % Boltzmann constant (eV/K)
e0=8.85e-14;  % permittivity of free space (F/cm)
q=1.602e-19;  % charge on an electron (coul)
KS=11.8;      % Dielectric constant of Si at 300K
ni=1.0e10;    % intrinsic carrier conc. in Silicon at 300K (cm^-3)
EG=1.12;      % Silicon band gap (eV) 

NA = 1e16;
ND = 1e19;

NDref = 1.3e17;
NAref = 2.35e17;
unmin = 92;
upmin = 54.3;
un0 = 1268;
up0 = 406.9;
an = 0.91;
ap = 0.88;

un = unmin + un0 / (1 + (ND/NDref)^an);
up = upmin + up0 / (1 + (NA/NAref)^ap);

Dn = k * T * un;
Dp = k * T * up;

tp = 1e-6;
tn = tp;

Ln = sqrt(tn * Dn);
Lp = sqrt(tp * Dp);

J0 = q * (Dn*ni^2/Ln/NA + Dp*ni^2/Lp/ND);


Vbi=k*T*log((NA*ND)/ni^2); 
xN=sqrt(2*KS*e0/q*NA*Vbi/(ND*(NA+ND)));    % Depletion width n-side
xP=sqrt(2*KS*e0/q*ND*Vbi/(NA*(NA+ND)));    % Depletion width p-side

W = xN + xP;

J = J0 * e^(VA / k / T) - J0
\end{lstlisting}

在 0.5, 0.6, -0.5, -0.6 分别得到 0.00069784,0.033400,-2.7784e-12,-2.7784e-12


\homep{5}

\subhome{a} 正偏,存在少子的积累问题。

\subhome{b} \(N_A = 1e16 cm^{-3}\)

\subhome{c} \(N_D = 1e14 cm^{-3}\)

\subhome{d} \(n_i^2 = 1e23 cm^{-6}, n_i = 3.1623e11 cm^{-3}\)

\subhome{e} \(np = n_i^2 \exp (q V_A/kT)\) 即 \(1e26 = 1e20 \exp (V_A / 0.026)\),\(V_A = 0.6 V\)

\homep{6}

\subhome{a} 正偏,空穴的能量相对变高了。

\subhome{b} 即费米能级的偏移,\(0.5 V\)

\subhome{c} 带隙为导带价带的差距,\(1.25 eV\)

\subhome{d} 内建电势为 \(0.25 eV\)


% \tableofcontents

% Start Here

% End Here

\end{document}