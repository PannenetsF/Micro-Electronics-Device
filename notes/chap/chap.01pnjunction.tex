\ifx\mainclass\undefined
\documentclass[cn,11pt,chinese,black,simple]{../elegantbook}
% 微分号
\newcommand{\dd}[1]{\mathrm{d}#1}
\newcommand{\pp}[1]{\partial{}#1}

% FT LT ZT
\newcommand{\ft}[1]{\mathscr{F}[#1]}
\newcommand{\fta}{\xrightarrow{\mathscr{F}}}
\newcommand{\lt}[1]{\mathscr{L}[#1]}
\newcommand{\lta}{\xrightarrow{\mathscr{L}}}
\newcommand{\zt}[1]{\mathscr{Z}[#1]}
\newcommand{\zta}{\xrightarrow{\mathscr{Z}}}

% 积分求和号

\newcommand{\dsum}{\displaystyle\sum}
\newcommand{\aint}{\int_{-\infty}^{+\infty}}

% 简易图片插入
\newcommand{\qfig}[3][nolabel]{
  \begin{figure}[!htb]
      \centering
      \includegraphics[width=0.6\textwidth]{#2}
      \caption{#3}
      \label{#1}
  \end{figure}
}

% 表格
\renewcommand\arraystretch{1.5}

% 日期

\newcommand{\homep}[1]{\Large\textbf{Problem #1}}
\newcommand{\subhome}[1]{\large\textbf{SubProblem #1}}

% MATLAB 代码块环境

% \usepackage{ctex}
\usepackage[numbered,framed]{matlab-prettifier}

\lstset{
  language = octave,
  style = Matlab-editor,
  basicstyle = \mlttfamily,
  escapechar = ",
  mlshowsectionrules = true,
}
% 
% \newenvironment{ocode}{\begin{lstlisting}[language=octave]}{\end{lstlisting}}

\lstnewenvironment{ocode}[1][]{%
    \lstset{language=octave,#1}}{}%

\begin{document}
\fi 
\def\chapname{01pnjunction}

% Start Here
\chapter{PN 结}

\begin{introduction}
    \item 什么是 PN 结
    \item 平衡态能带图
    \item 外加电压
\end{introduction}

PN 结是其他微电子器件的基础。

PN 结的常见应用有:太阳能电池,GaAs / GaN 激光器,有机发光二极管,雪崩光电二极管,CMOS 图像传感器。OLED 是有机半导体的显示材料,怕水,寿命有限。

\section{PN 结的形成}

PN 结最早是通过热扩散,目前有沉积、扩散、激光掺杂等工艺。

一般掺杂得到的是二维的器件,电流方向不是直线,难以求解。若是得到的是一个很窄的器件,可以简化为一个一维问题,对第二个维度的依赖性会变低,只需考虑第一维度的运动情况。

两种半导体直接相连是不能得到 PN 结的,因为断面上的原子不能形成化学键。对 N 型半导体,费米能级靠近导带,对于 P 型半导体,靠近价带。在两个体区中, N 型中很多的施主杂质电离的正离子(无法移动),但是有等量的电子(可以移动, \(n = N_D\));类似的 P 型存在可以移动的空穴(\(p = N_A\))。浓度差造成了扩散,那么 N 型靠近结区的部分电子被中和,整体带正电,P 型对应部分带负电,这就是耗尽区。由于扩散的存在,出现 \(n \cdot p > n_i^2\) 的瞬态。

电流分为两部分,漂移电流以及扩散电流。之前考虑的是扩散电流,因为此时还没有出现两部分的电势差。由于扩散后出现了静电荷,形成了电场,开始考虑漂移电流。电场逐渐增大,受到的阻力也越来越大,直到扩散电流与漂移电流相互抵消,进入稳态,即完成形成过程。

可以分为两个体区以及耗尽区(空间电荷区),耗尽区中没有自由移动的电荷, N 型一侧有 \(n \ll n_0\),另一侧有 \(p \ll p_0\) 。

\(E_c(x)\) 如\figref{01pnjunction:fig0201}

\qfig[fig0201]{0201.png}{\(E_c(x)\)变化}

\(n(x)\) 满足 \[n(x) = N_c \exp(-\frac{E_c(x)-E_F}{k T})\] 

其中, \(E_c - E_F \approx 0.06 eV\), \(k T \approx 26 meV\)。

依靠本式,得到,空间电荷区几乎没有自由电荷。最后的分布如\figref{01pnjunction:fig0202}

\qfig[fig0202]{0202.png}{PN 结区域}

形成的电场为均匀变化,最大值为 

\[E_{max} = \frac{q}{K_s \epsilon_0} x_p N_A = \frac{q}{K_s \epsilon_0} x_n N_D\]

可以得到对应的电势以及能带图,



% End Here

\let\chapname\undefined
\ifx\mainclass\undefined
\end{document}
\fi 