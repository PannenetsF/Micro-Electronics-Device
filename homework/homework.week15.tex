\documentclass[lang=cn,11pt,a4paper,cite=authoryear,twocolumn]{elegantpaper}

% 微分号
\newcommand{\dd}[1]{\mathrm{d}#1}
\newcommand{\pp}[1]{\partial{}#1}

% FT LT ZT
\newcommand{\ft}[1]{\mathscr{F}[#1]}
\newcommand{\fta}{\xrightarrow{\mathscr{F}}}
\newcommand{\lt}[1]{\mathscr{L}[#1]}
\newcommand{\lta}{\xrightarrow{\mathscr{L}}}
\newcommand{\zt}[1]{\mathscr{Z}[#1]}
\newcommand{\zta}{\xrightarrow{\mathscr{Z}}}

% 积分求和号

\newcommand{\dsum}{\displaystyle\sum}
\newcommand{\aint}{\int_{-\infty}^{+\infty}}

% 简易图片插入
\newcommand{\qfig}[3][nolabel]{
  \begin{figure}[!htb]
      \centering
      \includegraphics[width=0.6\textwidth]{#2}
      \caption{#3}
      \label{#1}
  \end{figure}
}

% 表格
\renewcommand\arraystretch{1.5}

% 日期

\newcommand{\homep}[1]{\Large\textbf{Problem #1}}
\newcommand{\subhome}[1]{\large\textbf{SubProblem #1}}

% MATLAB 代码块环境

% \usepackage{ctex}
\usepackage[numbered,framed]{matlab-prettifier}

\lstset{
  language = octave,
  style = Matlab-editor,
  basicstyle = \mlttfamily,
  escapechar = ",
  mlshowsectionrules = true,
}
% 
% \newenvironment{ocode}{\begin{lstlisting}[language=octave]}{\end{lstlisting}}

\lstnewenvironment{ocode}[1][]{%
    \lstset{language=octave,#1}}{}%


\title{微电子器件物理\quad 第十五周作业}
\author{范云潜 18373486}
\institute{微电子学院 184111 班}
\date{\zhtoday}

\begin{document}

\maketitle

\section{BJT}

\homep{1}

\(\beta_{dc} = 1.23 / 0.04 = 30.75\) 

\(\alpha_{dc} = 1.23 / 1.27 = 0.9685\)

\homep{2}

由于采用 npn 进行分析,课本基于 pnp ,将 n 和 p 互换。

\(\gamma = I_{En} / (I_{Ep} + I_{En}) = 1/1.005 = 0.99502\)

\(\alpha_T = I_{Cn}/I_{En} = 0.995\)

\(\beta_{dc} = I_c / I_b = 0.995 / 0.01 = 99.5\)

\(\alpha_{dc} = \gamma \alpha_T = 0.99005\)

\homep{3}

\subhome{a}

易知,在 C E 之中,电子为少子,因此是 p 掺杂,那么 这是一个 PNP 。

\subhome{b}

易得, BE 正向偏置, CB 反向偏置,据表,处于正向导通状态。

\subhome{c}

已知, \(N_B = 1.0 \times 10^{17} cm^{-3}\) 。

\[
\Delta n_{\mathrm{E}}\left(x^{\prime \prime}=0\right)=n_{\mathrm{E} 0}\left(e^{q V_{\mathrm{EB}} / k T}-1\right) = 1 \times 10^{11}
\]

由于 \(V_{EB}\) 的存在,需要另一个等式:

\[
\Delta p_{\mathrm{B}}(0)=p_{\mathrm{B} 0}\left(e^{q V_{\mathrm{EB}} / k T}-1\right) = 1 \times 10^{12}
\]

解得 \(n_{E0} = 0.1 p_{B0} = 100 \) ,那么 \(N_{E,A} = 1 \times 10^{18} cm^{-3}\) 。

\subhome{d}

\[
\gamma=\dfrac{1}{1+\dfrac{D_{E}}{D_{B}} \dfrac{N_{B}}{N_{E}} \dfrac{W}{L_{E}}} = \dfrac{1}{1 + 0.5 \cdot 0.1 \cdot 0.2} = 0.999
\]

\subhome{e}

\[
\beta_{\mathrm{dc}}=\dfrac{1}{\dfrac{D_{\mathrm{E}}}{D_{\mathrm{B}}} \dfrac{N_{\mathrm{B}}}{N_{\mathrm{E}}} \dfrac{W}{L_{\mathrm{E}}}} = 2 \cdot 10 \cdot 5 = 100
\]

\section{BJT 非理想}

1) B 2) A 3) E 4) B 5) D 6) A 7) C \footnote{是为了更好控制电流吗?}

% Start Here

% End Here

\end{document}