\documentclass[lang=cn,11pt,a4paper,cite=authoryear]{elegantpaper}

% 微分号
\newcommand{\dd}[1]{\mathrm{d}#1}
\newcommand{\pp}[1]{\partial{}#1}

% FT LT ZT
\newcommand{\ft}[1]{\mathscr{F}[#1]}
\newcommand{\fta}{\xrightarrow{\mathscr{F}}}
\newcommand{\lt}[1]{\mathscr{L}[#1]}
\newcommand{\lta}{\xrightarrow{\mathscr{L}}}
\newcommand{\zt}[1]{\mathscr{Z}[#1]}
\newcommand{\zta}{\xrightarrow{\mathscr{Z}}}

% 积分求和号

\newcommand{\dsum}{\displaystyle\sum}
\newcommand{\aint}{\int_{-\infty}^{+\infty}}

% 简易图片插入
\newcommand{\qfig}[3][nolabel]{
  \begin{figure}[!htb]
      \centering
      \includegraphics[width=0.6\textwidth]{#2}
      \caption{#3}
      \label{no :#1}
  \end{figure}
}

% 表格
\renewcommand\arraystretch{1.5}

% 日期

\newcommand{\homep}[1]{\section*{Problem #1}}
\newcommand{\subhome}[1]{\subsection*{SubProblem #1}}

% MATLAB 代码块环境

% \usepackage{ctex}
\usepackage[numbered,framed]{matlab-prettifier}

\lstset{
  language = octave,
  style = Matlab-editor,
  basicstyle = \mlttfamily,
  escapechar = ",
  mlshowsectionrules = true,
}
% 
% \newenvironment{ocode}{\begin{lstlisting}[language=octave]}{\end{lstlisting}}

\lstnewenvironment{ocode}[1][]{%
    \lstset{language=octave,#1}}{}%


\title{第一周作业}
\author{范云潜 18373486}
\institute{微电子学院 184111 班}
\date{\zhtoday}

\begin{document}

\maketitle

作业内容:1.1-1.45, 1.50, 1.59

\tableofcontents

% Start Here

\section{1.1-1.28}

\begin{itemize}
    \item 1.1: 5 CPU
    \item 1.2: 1 abstraction
    \item 1.3: 3 bit 
    \item 1.4: 8 computer family
    \item 1.5: 19 memory
    \item 1.6: 10 datapath(why not alu)
    \item 1.7: 9 control
    \item 1.8: 11 desktop or pc
    \item 1.9: 15 embedded system
    \item 1.10: 22 server 
    \item 1.11: 18 LAN 
    \item 1.12: 27 WAN 
    \item 1.13: 23 supercomputer
    \item 1.14: 14 DRAM
    \item 1.15: 13 defect
    \item i.16: 6 chip
    \item 1.17: 24 transistor
    \item 1.18: 12 DVD
    \item 1.19: 28 yield
    \item 1.20: 2 assembler
    \item 1.21: 20 os
    \item 1.22: 7 compiler
    \item 1.23: 25 VLSI
    \item 1.24: 16 instruction
    \item 1.25: 4 cache
    \item 1.26: 17 isa 
    \item 1.27: 21 semiconductor
    \item 1.28: 26 wafer 
\end{itemize}

\section{1.29-1.45}

\begin{itemize}
    \item assembler: i 
    \item cpp: b
    \item LCD: e 
    \item compiler: i 
    \item cray-1\footnote{https://en.wikipedia.org/wiki/Cray-1}: h
    \item DRAM: d
    \item IBM PC: f
    \item Java: b
    \item Scanner: c 
    \item MacOS: f 
    \item microprocessor: d
    \item ms word: a 
    \item mouse: c
    \item os: i 
    \item printer: e 
    \item silicon: g 
    \item spreadsheet: a
\end{itemize}

\section{1.46}

\[Time_{avg,rot,7200} = 0.5 / 7200 = 0.000069444 s\]

\[Time_{avg,rot,10000} = 0.5 / 10000 = 0.00005 s\]

\section{1.47}

在最外圈,一秒转 \(1/1600\) 圈;最内为 \(1/570\) ,那么分别存储 \(1600 \times 1.35 \times 10^6 = 2160000000 Byte\), \(570 \times 1.35 \times 10^6 = 769500000 Byte\)。

\section{1.50}



\section{1.59}

\section*{词汇}

wafer 晶片

yield 产率

% End Here

\end{document}