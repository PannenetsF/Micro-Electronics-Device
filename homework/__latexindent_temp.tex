\documentclass[lang=cn,11pt,a4paper,cite=authoryear]{elegantpaper}

% 微分号
\newcommand{\dd}[1]{\mathrm{d}#1}
\newcommand{\pp}[1]{\partial{}#1}

% FT LT ZT
\newcommand{\ft}[1]{\mathscr{F}[#1]}
\newcommand{\fta}{\xrightarrow{\mathscr{F}}}
\newcommand{\lt}[1]{\mathscr{L}[#1]}
\newcommand{\lta}{\xrightarrow{\mathscr{L}}}
\newcommand{\zt}[1]{\mathscr{Z}[#1]}
\newcommand{\zta}{\xrightarrow{\mathscr{Z}}}

% 积分求和号

\newcommand{\dsum}{\displaystyle\sum}
\newcommand{\aint}{\int_{-\infty}^{+\infty}}

% 简易图片插入
\newcommand{\qfig}[3][nolabel]{
  \begin{figure}[!htb]
      \centering
      \includegraphics[width=0.6\textwidth]{#2}
      \caption{#3}
      \label{no :#1}
  \end{figure}
}

% 表格
\renewcommand\arraystretch{1.5}

% 日期

\newcommand{\homep}[1]{\section*{Problem #1}}
\newcommand{\subhome}[1]{\subsection*{SubProblem #1}}

% MATLAB 代码块环境

% \usepackage{ctex}
\usepackage[numbered,framed]{matlab-prettifier}

\lstset{
  language = octave,
  style = Matlab-editor,
  basicstyle = \mlttfamily,
  escapechar = ",
  mlshowsectionrules = true,
}
% 
% \newenvironment{ocode}{\begin{lstlisting}[language=octave]}{\end{lstlisting}}

\lstnewenvironment{ocode}[1][]{%
    \lstset{language=octave,#1}}{}%


\title{微电子器件物理\quad 第八周作业}
\author{范云潜 18373486}
\institute{微电子学院 184111 班}
\date{\zhtoday}

\begin{document}

\maketitle

作业内容:

\tableofcontents

\part{IV 作业}

\homep{选择}

1) c 2) b 3) d 4) e 

\homep{计算1}

\subhome{a}

如 \figref{01} 。

\qfig[01]{h801.png}{半导体内静电势变化}

\subhome{b}

注意到电通量连续,因此电场有突变,如 \figref{02} 。

\qfig[02]{h802.png}{电场变化}

\subhome{c} 

是的,因为没有电流通过。

\subhome{d} 

\qfig[03]{h803.png}{空穴浓度变化}

\subhome{e}

\[\begin{aligned}
    p_{fb} &= n_i \exp{(E_i-E_F)/k_B T}\\
    &= 10^{10} e^{0.51 eV / 0.026 eV} = 3.3026 \cdot 10^{18}cm^{-3} 
\end{aligned}\]

\subhome{f}

\(p_{surface} = n_i = 10^{10} cm^{-3}\)

\subhome{g} 

\(\varphi_S = \Delta E_i = 0.51 eV\) 

\subhome{h}

% TODO: 交界处 E_i = E_F 

\(V_G = 0\) ,因为在交界后未产生能带弯曲。

\subhome{i}

根据 \(E_S = \left[\dfrac{2 q N_A}{K_S \epsilon_0} \phi_S\right]^{1/2}\) ,以及电通量关系 \(E_{ox} = \dfrac{K_S}{K_O}E_S\) ,得到 \(\Delta \phi_S = E_{ox} x_o \)

\[\begin{aligned}
    \Delta \phi_S &= x_o \frac{K_S}{K_O} \sqrt{\frac{2qN_A\phi_S}{K_S \epsilon_0}} \\
    &= \frac{x_o}{K_O \epsilon_0} \sqrt{2q N_A \phi_S K_S \epsilon_0} \\
    &= \frac{1.1e-7}{3.9 \cdot 8.8500e-14} \\
    &\sqrt{2 \cdot 1.6\: 10^{-16} \cdot 11.8 \cdot 8.85\: 10^{-14} \cdot 3.3 \: 10^{18} \cdot 0.51} \\
    &=   0.23916 V
\end{aligned}\]

\homep{计算2}

\subhome{a}

金属半导体没有功函数差,两侧费米能级对齐没有平带电压,\(V_{FB} = 0\) 。

\subhome{b} 

\[V_G = \phi_S + \Delta \phi_{ox}\] 

\[\Delta \phi_{ox} = E_{ox} x_o \]

\[E_{ox} = \dfrac{K_S}{K_O}E_S\]

\[E_S = \left[\dfrac{2 q N_A}{K_S \epsilon_0} \phi_S\right]^{1/2}\] 

联立 

\[\begin{aligned}
    V_G &= \phi_S + 0.30292 \sqrt{\phi_S} = 1 \\
    \phi_S &= 0.86^2 = 0.7396 V
\end{aligned}\]

\subhome{c and d}

在 (b) 中已经计算过,\[E_S = 782733.54786 V/cm\] ,\[E_{ox} = 3.0256 E_S = 2368270.73456 V/cm\]。

\subhome{e}

耗尽区宽度满足

\[W=\left[\frac{2 K_{\mathrm{S}} \varepsilon_{0}}{q N_{\mathrm{A}}} \phi_{\mathrm{S}}\right]^{1 / 2} = 0.0000018898 cm\] 

\subhome{f}

\[Q_{B}=-\sqrt{2 q \kappa_{S} \varepsilon_{0} N_{A} \phi_{s}}\]

\[Q_S = Q_B =  -0.00000081741 C/cm^2\]

\subhome{g}

由电荷平衡 \(Q_G = 0.00000081741 C/cm^2\)

\subhome{h}

\[\Delta \phi_{ox} = E_{ox} x_o = 0.26051 V\]

\subhome{i}

\[V_{\mathrm{T}}=2 \phi_{\mathrm{F}}+\frac{K_{\mathrm{S}} x_{\mathrm{o}}}{K_{\mathrm{O}}} \sqrt{\frac{4 q N_{\mathrm{A}}}{K_{\mathrm{S}} \varepsilon_{0}} \phi_{\mathrm{F}}}\]

\[\phi_F = \frac{k_B T}{q} \ln {\frac{N_A}{n_i}} =  0.50476 V\]

带入: \(V_T = 2 \cdot  0.50476 + 0.30436 = 1.3139 V\) 

\homep{计算3}

\subhome{a}

夹断之前

\[I_{\mathrm{DS}}=\frac{Z \bar{\mu}_{\mathrm{n}} C_{\mathrm{o}}}{L}\left[\left(V_{\mathrm{GS}}-V_{\mathrm{T}}\right) V_{\mathrm{DS}}-\frac{V_{\mathrm{DS}}^{2}}{2}\right]\]

修改为

\[I_{\mathrm{DS}}'=\frac{Z \bar{\mu}_{\mathrm{n}} C_{\mathrm{o}}}{L}\left[\left(V_{\mathrm{GS}}'-V_{\mathrm{T}}\right) V_{\mathrm{DS}}'-\frac{V_{\mathrm{DS}}'^{2}}{2}\right]\]


饱和后

\[I_{\mathrm{Dsat}}=\frac{Z \bar{\mu}_{\mathrm{n}} C_{\mathrm{o}}}{2 L}\left(V_{\mathrm{G}}-V_{\mathrm{T}}\right)^{2}\]

修改为

\[I_{\mathrm{Dsat}}'=\frac{Z \bar{\mu}_{\mathrm{n}} C_{\mathrm{o}}}{2 L}\left(V_{\mathrm{GS}}'-V_{\mathrm{T}}\right)^{2}\]

其中 \(V_{GS}' = V_{GS} - (V_S + I_{DS}' R_S)\) , \(V_{DS}' = V_D - I_{DS}' R_D  - (V_S + I_{DS}' R_S)\)

可以预见的将会修正为四次方程。

\subhome{b}

线性区


\[I_{\mathrm{DS}}=\frac{Z \bar{\mu}_{\mathrm{n}} C_{\mathrm{o}}}{L}\left[\left(V_{\mathrm{GS}}-V_{\mathrm{T}}\right) V_{\mathrm{DS}}\right]\]

修正 

\[I_{\mathrm{DS}}=\frac{Z \bar{\mu}_{\mathrm{n}} C_{\mathrm{o}}}{L}\left[\left(V_{G} - (V_S + I_{DS}' R_S)-V_{\mathrm{T}}\right)(  V_D - I_{DS}' R_D  - (V_S + I_{DS}' R_S))\right]\]

显然, \(R_S\) 或者 \(R_D\) 越大,线性区电流越小。在饱和区, \(R_D \) 影响忽略,\(R_S\) 越大,电流越小。

\part{非理想}

\homep{选择}

1) d 2) b 3) b 

\homep{解答}

\subhome{a}

\[\Delta V_G = - \frac{Q_F}{C_O} = -\frac{Q_F}{\frac{K_O \epsilon_0}{t_{O}}} = - \frac{1e11 q}{0.0000023600} = 0.0067881 V \]

引起负的偏移。

改变阈值的原理:在氧化层的电容中引入电荷,进而改变经过栅的电压降落。

\subhome{b}

\[\Delta V_G = - \frac{Q_F}{C_O} = \]




% Start Here

% End Here


\end{document}